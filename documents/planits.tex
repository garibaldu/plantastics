\documentclass[12pt]{article}
\input{LocalStyle.sty}  

\title{just thinking about doing slamnastics}
\author{tim, marcus}
\date{}
\begin{document}
\maketitle

\section{a PGM}

Here's what we think a probabilistic graphical model might look like:

\begin{figure}[h]
  \centering
  \includegraphics[scale=1.]{pics/megaSLAM_PGM}
  \caption{PGM for thinking about our megaslamisms.}
\end{figure}

\begin{itemize}
\item $\alpha$ is a prior over worlds
\item $\bx$ is the actual world
\item $\theta_t$ is a position and pose of the camera (in some coordinate frame)
\item $\by_t$ is an image (a view if you like), that includes depth.
\end{itemize}

The system has an initial assumed viewpoint, $\theta_0$.

We then get a series of views of parts of the world, and although we
don't know the exact viewpoints we know roughly how they're likely to
change.

We know how viewpoint combines with the world to create an view.


\section{simplest possible world that exposes the core inference problem?}

What about the world being one-dimensional, discretized in space, and
each spatial location having a ``height''? A camera detects these
heights (plus noise) but only within a window of fixed width around
its current position. The camera's position changes smoothly (meaning
small changes only) over time but the actual position change is
unknown (must be infered from the image).

\begin{itemize}
\item $\theta_t$ is just a scalar: the position of the camera. It always points straight down, and detects true vertical depth within its field of view.
\item an observation $y_t$ is a vector of depth values, $\by_t =
  (x_{\theta-d}+\epsilon : x_{\theta+d}+\epsilon)$ with the $\epsilon$
  all i.i.d. Gaussian.  That is, the observations are noisy versions
  of the true depth, $x_i$, over just a finite window.
\end{itemize}

Building a map means correctly representing the true depths and their
correct locations, ugh, that's not right............

\section{inference}

\subsection{exact is infeasible}
Can one model map uncertainty ``properly'', by having a prior
and updating to a general posterior? No, because
  \begin{enumerate}
  \item maps are very high-dimensional and the posterior is not going
    to factor (ie. the dimensions will be highly correlated), so full
    inference of this kind is hopelessly intractable;
  \item the posterior might also be massively multimodal due to
    non-identifiability (exchangability of latent variables). Not sure
    this is an issue though so I'm just noting it....
  \end{enumerate}

\subsection{representing a posterior over maps in some other way}




\end{document}
